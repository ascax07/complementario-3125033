\documentclass[12pt,a4paper]{article}
\usepackage[utf8]{inputenc}
\usepackage[spanish]{babel}
\usepackage{graphicx}
\usepackage{amsmath}
\usepackage{hyperref}

\title{Oso de la Monta\~na: Una Soluci\'on Tecnol\'ogica para la Gesti\'on de Cafeter\'ias}
\author{Jose Manuel Gasca Bonilla \\
Camilo Andr\'es Bautista}
\date{}

\begin{document}

\maketitle

\begin{abstract}
El proyecto "Oso de la Monta\~na" presenta un software innovador dise\~nado para optimizar la gesti\'on de productos, mesas y \'ordenes en cafeter\'ias. Este sistema permite a los meseros tomar \textit{\'ordenes} mediante una aplicaci\'on m\'ovil, sincroniz\'andolas en tiempo real con la administraci\'on central del negocio. Utilizando metodolog\'ias \textit{\'agiles} y tecnolog\'ias modernas como React Native y Firebase, este proyecto aborda las deficiencias operativas comunes en cafeter\'ias, mejorando la eficiencia y la experiencia del cliente. Los resultados preliminares indican una reducci\'on significativa en los tiempos de espera y un aumento en la satisfacci\'on de los clientes.
\end{abstract}

\section{Introducci\'on}
El sector de la hosteler\'ia enfrenta retos constantes relacionados con la eficiencia operativa, especialmente en cafeter\'ias donde la gesti\'on manual de pedidos y productos puede generar errores y retrasos. El proyecto "Oso de la Monta\~na" surge como una respuesta a estas problem\'aticas, proporcionando una soluci\'on integral para administrar productos, mesas y \textit{\'ordenes} de manera \textit{\'agil} y eficiente. Este software busca mejorar los procesos internos y ofrecer una experiencia de alta calidad tanto para los empleados como para los clientes.

\section{Metodolog\'ia}
El desarrollo del proyecto utiliz\'o la metodolog\'ia \textit{\'agil} Scrum, dividiendo el trabajo en sprints cortos y enfocados. Esto permiti\'o una adaptaci\'on r\'apida a los cambios y una comunicaci\'on constante entre los integrantes del equipo. Las herramientas clave incluyeron:
\begin{itemize}
    \item React Native: Para desarrollar una aplicaci\'on m\'ovil multiplataforma intuitiva y funcional.
    \item Firebase: Utilizado para la autenticaci\'on, la sincronizaci\'on en tiempo real y el almacenamiento de datos.
    \item Firestore: Base de datos escalable para gestionar la informaci\'on de productos, \textit{\'ordenes} y mesas.
    \item Figma: Para dise\~nar interfaces gr\'aficas que mejoren la experiencia del usuario.
\end{itemize}
El equipo de desarrollo organiz\'o las tareas mediante la plataforma Trello, lo que permiti\'o un seguimiento detallado de las actividades y garantiz\'o la entrega oportuna de cada funcionalidad.

\section{Resultados}
El sistema "Oso de la Monta\~na" cuenta con las siguientes funcionalidades principales:
\begin{enumerate}
    \item Gesti\'on de productos: Permite agregar, modificar y eliminar elementos del men\'u.
    \item Administraci\'on de mesas: Facilita el monitoreo en tiempo real de la disponibilidad de mesas.
    \item Toma de \textit{\'ordenes}: Los meseros pueden generar pedidos a trav\'es de la aplicaci\'on m\'ovil, asign\'andolos a mesas espec\'ificas.
    \item Sincronizaci\'on en tiempo real: Las \textit{\'ordenes} se reflejan autom\'aticamente en el sistema central, acelerando su preparaci\'on.
\end{enumerate}
En pruebas piloto, el software logr\'o reducir en un 25\% los tiempos de espera promedio para la toma y entrega de pedidos. Asimismo, la satisfacci\'on del cliente aument\'o en un 15\%, seg\'un encuestas realizadas.

\section{Discusi\'on}
El proyecto "Oso de la Monta\~na" demuestra c\'omo las soluciones tecnol\'ogicas pueden transformar la gesti\'on de cafeter\'ias. La implementaci\'on de sincronizaci\'on en tiempo real y una interfaz intuitiva permitieron abordar las ineficiencias comunes en el sector. Sin embargo, se identificaron \'areas de mejora, como la incorporaci\'on de funcionalidades adicionales (p. ej., pagos en l\'inea) y la capacitaci\'on m\'as extensa del personal.

\section{Conclusiones y Recomendaciones}
En conclusi\'on, "Oso de la Monta\~na" ofrece una soluci\'on integral para la gesti\'on de cafeter\'ias, optimizando los procesos operativos y mejorando la experiencia del cliente. Los resultados obtenidos respaldan la eficacia del sistema y su potencial para ser implementado en establecimientos similares.

Para futuras iteraciones del proyecto, se recomienda:
\begin{enumerate}
    \item Expandir las funcionalidades: Incorporar m\'odulos para pagos en l\'inea y reportes anal\'iticos.
    \item Desarrollar una versi\'on para clientes: Que permita realizar pedidos desde dispositivos m\'oviles.
    \item Ampliar el alcance: Implementar el sistema en diferentes tipos de negocios del sector de alimentos y bebidas.
\end{enumerate}

\end{document}
